\documentclass[../main.tex]{subfiles}

\begin{document}


\begin{abstract}

Il sole \'e una massa di gas autogravitante che supporta numerosissimi modi di oscillazione attorno alla sua posizione di equilibrio. Il loro studio fornisce uno strumento per determinare caratteristiche essenziali della struttura solare. In seguito alla comprensione della struttura spettrale delle oscillazioni solari sono state identificate altre stelle che mostrano oscillazioni con analoga struttura.

La rivelazione dei moti periodici della fotosfera solare (oscillazione dei 5 minuti: \cite{lei62velocity}) e la scoperta, in misure in cui la superficie solare \'e risolta spazialmente (\cite{deu75observations}) e in misure integrate sull'intero disco solare (\cite{cla79solar}), che tali moti periodici sono la sovrapposizione di modi discreti, sono le basi osservative dell'eliosismologia: quest'ultima studia le oscillazioni della superficie solare e le informazioni sulla struttura interna in esse contenuta.

{\itshape le frequenze sono determinate principalmente dalla stratificazione e dinamica della regione in cui le ampiezze delle oscillazioni sono apprezzabili.}

\'E possibile calcolare numericamente le frequenze di oscillazione attese sulla base di un modello stellare e al variare di uno o pi\'u parametri del modello analizzare la corrispondenza con quelle osservate inoltre sono state sviluppate tecniche per analizzare il problema inverso e valutare le lacune del modello solare dalla differenza tra frequenze predette e osservate.

Scrivo le leggi di conservazione che determinano la struttura stellare e la sua evoluzione: le oscillazioni di periodo attorno a 5 minuti sono descritte come piccole perturbazioni adiabatiche dello stato di equilibrio. Considero le problematiche generali che riguardano la costruzione di un modello solare con particolare riferimento alla determinazione dell'abbondanza iniziale di $\chem{^4He}$.

Descrivo brevemente le osservazioni relative  alle oscillazioni con periodo 5 minuti e la loro struttura modale, che sar\'a giustificata tramite il modello proposto da \citet{ulrich70five} e \citet*{stein71five}, quindi descrivo le tecniche di analisi del campo di velocit\'a della superficie solare.

Introduco le perturbazioni lineari adiabatiche attorno allo stato di equilibrio idrostatico descrivendo le perturbazioni della densit\'a, pressione ed energia potenziale gravitazionale tramite una pulsazione $\omega$ e, per la dipendenza spaziale, tramite ampiezza radiale e armonica sferica $Y_{lm}(\theta,\phi)$: ottengo le equazioni che determinano i modi di oscillazione discreti ordinati, per l fissato, tramite l'ordine n crescente con frequenza e numero di nodi radiali e l'ampiezza della perturbazione.

\'E possibile trascurare la perturbazione del potenziale gravitazionale entro un'accuratezza delle frequenze dei modi $\frac{\Delta\omega}{\omega}\approx0.01$: ottengo, nel limite di alte e basse frequenze, la relazione di dispersione per onde acustiche e di gravit\'a e un'espressione analitica per le frequenze dei modi di oscillazione.

\begin{comment}
Accenner\'o brevemente alle tecniche osservative ed alle problematiche legate alla precisione richiesta dalle osservazioni eliosismologiche.
Ricavo il sistema di equazioni differenziali che descrive le perturbazioni adiabatiche e scrivo la relazione di dispersione nella forma pi\'u generale. Da calcoli accurati risulta che i modi p sono confinati nella parte esterna della zona convettiva.
Introduco quindi i modi normali per il moto ondoso. Suppongo che le grandezze fisiche che determinano il problema dipendano solo dalla distanza dal centro, \'e quindi naturale descrivere l'ampiezza delle perturbazioni in termini di armoniche sferiche per la dipendenza angolare che sono identificate dalla distribuzione caratteristica delle fasi di oscillazione sulla superficie solare. Queste definiscono il grado l del modo normale; le autofunzioni dell'ampiezza dell'oscillazione radiale sono caratterizzate dall'indice n, il cui modulo riflette il numero di zeri dell'ampiezza radiale.
La piccola ampiezza delle oscillazioni permette di usare la teoria delle perturbazioni lineari applicata alle equazioni di un un corpo autogravitante in equilibrio idrostatico per ricavare attraverso l'equazione del moto (equazione di Eulero) un'equazione agli autovalori per le frequenze di pulsazione. Questo tipo di problema \'e comune in fisica teorica quindi esistono molte tecniche numeriche per determinare le frequenze con l'accuratezza necessaria per il confronto con i dati sperimentali ma non tratter\'o questa problematica; descrivo il comportamento asintotico per alte/basse frequenze delle oscillazioni adiabatiche trascurando la perturbazione del potenziale gravitazionale.
I risultati eliosismologici dimostrano la validit\'a dei modelli solari standard di cui descrivo le caratteristiche fondamentali; di particolare importanza \'e la  misura  dell'abbondanza di elio nel Sole, valore che nei modelli solari standard viene variato, insieme al parametro che regola l'efficienza del trasporto energetico nella zona convettiva, per ottenere, determinando numericamente l'evoluzione del modello iniziale, i giusti valori di luminosit\'a e raggio attuali.
Le frequenze dei modi normali dell'interno solare contengono informazioni sul profilo radiale della velocit\'a del suono, densit\'a, accelerazione di gravit\'a ed esponente adiabatico $\Gamma_1$:
Infine mostro come l'inversione eliosismologica fornisca una guida per individuare le zone in cui il modello solare appare non completamente corretto.
\end{comment}

La soluzione del problema inverso permette, tramite un'osservazione accurata delle frequenze solari, di ricavare correzioni al modello solare e limitare il range dei parametri del modello.

Ricavo il profilo radiale della velocit\'a del suono in maniera analitica ma afflitta da errori sistematici che \'e possibile mitigare considerando invece le differenze tra frequenze calcolate tramite un modello solare e osservate: in quest'ultimo caso \'e possibile correggere la profondit\'a della zona convettiva.

Introduco le tecniche di inversione non asintotica che si basano sulla soluzione numerica delle equazioni delle oscillazioni da cui ottengo correzioni al profilo radiale di $\rho$ e $c_s$ e la volocit\'a di rotazione del Sole supposta a simmetria sferica.

\end{abstract}


\end{document}